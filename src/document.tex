\documentclass[12pt]{report}

\usepackage[margin=1.0in]{geometry}

\usepackage[
backend=biber,
style=numeric,
sorting=none
]{biblatex}
\addbibresource{references.bib}

% Title Page
\title{Robinson Observatory Telescope Refresh}
\author{Bradley Glaser, Derek McCrae, Bryan Ocbina, Justin Rudisal}


\begin{document}
\maketitle

\begin{abstract}
\end{abstract}

\section*{Executive Summary}
Do what you say you will do.\cite{heinrich}

\section*{Technical Content}

\subsection*{Project Narrative}

The Robinson Observatory is a facility at UCF that is used for both education and research on the topics of astronomy.  It is run by members of the Planetary Sciences Group and the Astronomy Society in the physics department.  The observatory provides high resolutions of astronomical objects and various events and programs involving the UCF and Central Florida to connect members of the community to astronomy .  The main telescope at the observatory is a 20-inch f/8.2 RCOS Ritchey-Chretien telescope and has recently had issues regarding its mobility.  The telescope is able to align itself to a starting reference point, but when given instructions to move to a certain coordinate, the telescope will unreliably move towards the coordinate without any relation to its physical capabilities or location.  This leads to a layer of issues.  The main issue involves being completely off-course of the desired coordinate.  The telescope software will either output warnings too early or too late and not prevent the telescope from colliding with its mount which causes damage to the entire system.  This is a hazard to the staff and the powerful and high cost telescope itself.

The origin of the problem is questioned whether it is a mechanical issue or a problem with the various software that are involved in operating the telescope.  Our computer science team alongside the electrical engineering and mechanical engineering teams will work together to determine the issues and provide solutions that will assist in leading bringing back the telescope to full functionality.  The main concern of the team at the observatory in regards to computer science is that the software of all of the components of the telescope are able to communicate properly on Windows 10.  The components are the telescope, TheSkyX Professional software, the focuser software, and the dome control software.  Meeting this specification will require research into each component of the system and also research into possible software/driver compatibility issues.  The knowledge to code drivers ourselves may be required.  Possible roadblocks may be that certain softwares are not updated or maintained to current standards which would require an overhaul of the system to have all components compatible.  Having the telescope and its surrounding system compatible will be the main objective of our project.

The second objective for our team will be to assist the research opportunities  and community outreach events available at the Robinson Observatory.  Our team will be designing a database capable of storing the images, and their metadata, taken directly by the telescope.  This database will function alongside a user-friendly website that can be used by the community for sharing images and also by the astronomer team to conduct research.  The images will be indexed with their relevant data and a search functionality will be available.  This will provide storage for the data collected at the observatory and assist in long-term research.  On the community side, the website will allow the observatory to easily share images with those interested.  A possible stretch goal is to use machine learning to search images for known objects based on a gathering of previous tagged images of the object.

The third objective will be to assist the electrical engineering team in modeling the black box which serves as the main component for issuing commands to the telescope.  The model will replicate the functionality of the telescope in turns of movement and sending and receiving data.  The current goal for the model is to be able to track the sun or moon within the accuracy of a degree.  This model will assist future teams as the complexity of the black box has been deemed to be too large in scope for a single year senior design team.  The reverse engineering of the black box will be useful for moving away from proprietary systems and give greater control to the team at the observatory in terms of what they would like out of the telescope, making it more programmable to their personal needs.

There are also a series of additional objectives pertaining to recording and storing astronomical data that will be obtained from the software when taking images.  They are outlined in the later sections of this document but serve to either allow for better research opportunities for the astronomer team or to simplify a process and improve the functionality of the facility through accessibility between systems at the observatory or allowing for online access to the faculty or community.

The refresh of the Robinson Observatory is a goal that is extremely important to each member of our team.  This project serves as an opportunity to give back to our university for all the experiences we have gained in our past few years here.  A reminder of all the knowledge we have accumulated into a single project demonstrating our technical skills and problem solving ability.  We hope to be able to remember our senior design project proudly as we bring an end to our time with education and proceed into the professional world of computer science.  Each of our members would also like to address their own personal interests and motivations in this project.

\subsection*{Personal Interest}

\subsubsection*{Justin Rudisal}

My personal interest in this project is twofold with the first part being due to the subject area being space, and the second due to the project area involving a level of mechanical integration. Within my own personal experience with Internships, I have worked heavily with website development and database management. However, I have absolutely no knowledge of mechanical systems and how a computer communicates with a physical, moving part. My dream goal is working in the creative entertainment programming field, such as Universal Creative or Disney Imagineering, and so I need this basic machine programming experience to be successful in my pursuit of a career in those areas.

I also have been in love with space and space exploration for as long as I can remember, and so much so that I originally was planning on doing a double major in both Computer Science and Astronomy until I sadly realized I could not afford to pursue both. So when I heard of the senior design project that would be combining both the area of knowledge that I wish to learn more in with an area of passion that I love, I knew I had to jump on it as quickly as I could.

Other major considerations of mine for this project include the fact that it is a completely open-source undertaking. The end goal of this is to be able to document and demonstrate everything we do so that other small observatories around the world can easily duplicate our results. Projects such as that are the ones that drive my passion in programming, as well as the engineering field in general, because I fully stand behind and support anything with altruistic goals that benefits other people for the common good. I personally believe intellectual progress is achieved from the sharing of knowledge freely, and not from the selling of it for a profit.

I also want to leave a legacy behind me within my university that I have called home for the past handful of years, and that has brought so many different opportunities into my life. I want to give back to the place that has given so much to me. Being able to refresh the telescope and have my name on such a substantial undertaking is one way that I can give back and leave that legacy. I can come back years later, point at the observatory, and say "I helped repair that and bring it into the modern era". Now that is both a great resume builder, and overall achievement to be proud to talk to others about.

I remember one of my most favorite experiences from my first years at college was actually going to the observatory for "Knights Under the Stars" throughout my first few semesters, and being able to watch the telescope work and see the screen show images of our galaxy. I thought that was one of the most fascinating things ever, and it sparked an even bigger passion in astronomy in my life. Lack of time, and it's breaking, caused me to stop going as I got into my higher level classes, but that initial impression left on me caused me to instantly want to help repair the observatory when it was announced as a senior design project.

\subsubsection*{Derek McCrae}

I want to embark on this project primarily for my interest in space. Growing up near the Kennedy Space Center, I was viewing the shuttle launches from a young age and dreamed upon what is out there. A sight I will always want to see firsthand has been strengthened by the pictures provided by telescopes. Telescopes provide a great photos of our universe that can't be currently reached by humans like the Hubble Telescope showcasing the Pillars of Creation or a simple home telescope to view Jupiter. This passion that has grown stronger with the advancements being made for commercial trips to space, especially the advancements by SpaceX. The technological advancements SpaceX has made with getting their boosters to land safely back at a landing pad simultaneously is one of the coolest things I have seen in the last decade. SpaceX along with Virgin Galactic have brought excitement back to space exploration. Some great memories I have include watch parties that had telescope viewings of either a launch, or eclipse, or viewing of another planet.

I had two events that greatly enhanced my passion for space. The first one happened when I was in 7th grade and our class trip was to the Kennedy Space Center. We were able to tour the facilities and sleep under one of the rockets that had on display. The trip helped gain new knowledge and greater appreciation of space. The second event happened freshman summer of high school on a trip to North Carolina. I was family on the side of a mountain in some cabins. One night the mountain staff held a viewing party. My family along with other guests were able to view the stars and use a telescope to see planets only viewable at that time of the year. The event showed me how the topic of space can bring people together.

These two events are a cornerstone to why I wanted to be apart of this project. I know firsthand how viewing parties can bring people together and the knowledge can be beneficial. Not many people own their own personal telescope so this observatory can give those interested a better way to view the sky. To know, I will have a helping hand in allowing activities to once again be performed at the observatory is exciting. The observatory can bring back “Knights Under the Stars” allowing many the current unavailable opportunity of getting to see the stars up close. The observatory could even provide new research study opportunities for those at UCF.

The next reason I chose this project was to build experiences on ideas I have yet to work with. My initial interpretation for this project was a new software needs to be created to help the telescope return to working order. I have learned how to code basic programs and taught programming stepping stones over the past few years without getting any real world experience. I have been eager to apply my knowledge to develop something that will be used past my educational career here at UCF. Furthermore, it will be applying this knowledge to help build something in an industry I am passionate about.

My final motivation for this project is the ability to leave behind a legacy. Of course, the legacy could become poor if this project isn’t completed properly but with proper completion, our solution will outlast my time here at UCF. This refresh could impact this university and space exploration for the next few years and that provides a strong legacy.

\subsubsection*{Bradley Glaser}

I have always had a deep love for everything space. Likewise, my science classes were always some of my favorite courses in school. From a very young age I was always asking questions and some of the biggest unanswered questions are extraterrestrial. The exploration and study of the cosmos must be one of humanity’s priorities if we are to advance as a civilization. I realize that my contributions to the Robinson Observatory are not quite earth shattering. However, I am happy to be able to say that I am a small part of that undertaking.

My father was always a very strong influencer in my life and he has always pushed me to pursue my goals. When I was in elementary, he got me my first childhood telescope. It was one of my most prized possessions. I kept it for many years and I have fond memories of stargazing with my friends and family. This project is an amazing opportunity to work on a full size telescope. One the day that we were assigned the projects, I called my father and told him about my selection. He was over the moon about it. I cannot wait to dig into this project and make it a reality.

My fiance and I have a special connection because of stargazing. When we lived in Utah, some of our earliest dates were spent on the top of a mountain looking at the stars and moon. We would hike up the mountain as the sun went down and then spend our time laying on the ground and watching for shooting stars. Our relationship was built on a shared love of science and space. Space has been a big influence on my life and I hope that my involvement in this project will help others to capture that same magic.

Intellectual property is a big concern of mine. I understand the value of having outside companies come to UCF and pitch projects. Students need real work experience and those businesses are able to provide it. However, the cynical side of me sees these businesses profiting off of the work senior design students are performing for them. Thus, I wanted to choose a project that would be internal to UCF. That way I can be sure that my time spent will not just be putting money into someone else’s coffers.

I wanted to work on a project that was tangible. Many of the projects presented for us to pick from were either esoteric or difficult to explain to non technical persons. Explaining that I will be on the team that fixes UCF’s observatory is immediately easy for anyone to understand. Likewise, I felt that potential employers would be more responsive to a project which was for the immediate betterment of others. It is hard for the average person to understand how “analyzing blockchain correctness” will affect them. However, I am able to point to the telescope and say “you couldn’t visit that before and now it works”. That direct impact is something that I wanted to have for a project.

Being able to contribute to UCF and help future students foster their love of astronomy is a dream come true. I hope that my contributions to this project will help others discover their passions. Similarly, the ability to leave a legacy behind at UCF is a big bonus. Many of the projects presented to us seemed fleeting or shallow in my opinion. They dealt with trivial things that, in the case of blockchain, might not even be relevant in the coming years. I hope that our contributions are more substantial and that they will stand the test of time.

\subsubsection*{Bryan Ocbina}

After hearing the pitch for this project, I realized being able to provide a meaningful impact on the UCF community was significant for myself.  I have grown up in Orlando all my life and have been ingrained to love all thing related to Florida from Disney and mickey mouse to the beaches and manatees to watching rocket and shuttle launches liftoff from Kennedy Space Center, with pure amazement every single time.  As I enter my final year as a Computer Science student, I would like to create a lifelong experience and memory here in my hometown before I finish my education, graduate, and begin to use my knowledge and technical skills in the professional world wherever my career may take me.  Aiding in the Robinson Observatory refresh is a project that aligns with my interests and goals and is my first choice amongst the about fifty projects presented to us.

One of my favorite memories is going on a camping trip a few summers ago and being to stargaze with close to zero light pollution.  The night sky at this time was and is still the most amazing and unforgettable sight I have ever seen.  I had a realization after seeing the sheer number of stars and the surprising amount of color of how much beauty we miss out on living in the city with lights everywhere.  Likewise, as another unforgettable experience was a viewing of a nighttime shuttle launch from my home in Orlando that lit up the sky as if the sun was rising.  Space and the stars above bring out childlike wonder in myself and brings excitement as our team begins to develop our design for the Robinson Observatory refresh.  To be able to once again share images from the telescope at the observatory and hopefully inspire some childlike wonder in someone else is an important part of this undertaking.

The UCF community and Central Florida as a whole is a wonderful center for growth and dreams.  Community outreach is a valuable aspect of this project.  Public events such as “Knights Under the Stars” held at the Robinson Observatory allow for people of all ages to explore the night sky and learn about our solar system and various nearby stars or other celestial phenomenon.  However with the main 20-inch telescope out of commission, the ability to view the night sky at these events is undoubtedly weakened.  We hope to not only restore this functionality but to also add on and allow for images from the 20-inch telescope to be shared more easily with the community.

As another part of community outreach, a decisive part of joining this project includes the open-source aspect of working with UCF.  This project serves to better the research opportunities available locally in Central Florida and our work done with the observatory will not be used to profit some business’s latest product.  The story of the project is much more involved and personal than the other projects pitched to our class.  We are able to not just present a fleshed out, functional telescope as a project we worked on but as an entire experience that could serve to inspire a future astronomer, aerospace engineer, or astronaut.

This project will be my number one priority for my remaining time as a student and I hope to gain as much as I can while working with everyone at the observatory and the other student teams as we share our experiences of our different fields and do our best to create an unforgettable experience.  Altogether, the Robinson Observatory refresh project serves as a platform to express my sincere appreciation for my hometown Orlando, Florida. I have many hopes for my senior design team that we will be able to accomplish some amazing features that will modernize the observatory and that will be used and remembered for many years after we have all graduated.  To be able to come back home, visit the observatory, and tell others that I worked on this feature or component at the Robinson Observatory would be a wondrous feeling.

\subsection*{Broader Impacts}

Getting the telescope and observatory working will help with both the different organizations that contract out the observatory for their own scientific pursuits, as well as enhancing its ability to reach out to students and the surrounding community with an accessible means to the stars. Even those who are not local will be able to benefit from this project due to one of our goals being to create an online public viewing platform for what the observatory is currently observing and completing. This opens it up to anyone with access to an online device to actively engage with the observatory, as well as the scientific space community at large. Our goal is to create both an online portal for the observatories sponsors to access the device, as well as crafting a learning environment for everyone.

This online outreach will also help spearhead engagement within space exploration and space overall within STEM groups both in the university as well as local elementary, middle, and high schools. Teachers will be able to pull up the current observatory pictures and developments, and will also be able to speak on weekly events that the observatory will be hosting. The overall goal is to both repair the observatory and bring it up to operating standards, as well as revamping its public outreach through new websites.

The organization that runs the observatory, Florida Space Institute, is a research group that is funded completely off of scientific grants and personal donations. By repairing this observatory for the Florida Space Institute we are helping an organization that may otherwise have a tough time affording the hiring of professionals to accomplish the intended goals. We are also helping to free up the time of those within the organization who may have had to spend their own time figuring out complicated temporary workarounds for the observatories problems instead of doing the scientific work that they want to accomplish.

\printbibliography

\end{document}
